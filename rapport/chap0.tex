\chapter*{Introduction}
Ce rapport présente l'avancée du projet fil Rouge 2018-2019, que notre équipe projet, \textbf{Charles Thérond}, \textbf{Ioan Catana}, \textbf{Karine Pétrus} et \textbf{Stéphane Mulard}, réalisé dans le cadre du Mastère Spécialisé BigData de Telecom ParisTech, pour l'entreprise Engie et qui consiste à \textbf{détecter les sous-performance des équipements photovoltaïques} en analysant les empreintes des différentes perturbations du signal électrique. Ce projet est encadré par \textbf{Stéphan Clémençon}, responsable du Mastère BigData pour Télécom ParisTech, et par \textbf{Paul Poncet}, Data Science Lead pour Engie.

Depuis le démarrage du projet en novembre 2018 nous avons pu en cerner les contours, affiner la problématique, explorer les approches méthodologiques existant dans la littérature scientifique, réaliser des premiers tests et définir une feuille de route pour les mois à venir, jusqu'à la soutenance finale en juin 2019.

Cette première version du rapport rend compte de notre travail à la fin janvier 2019. Dans un premier temps nous présentons le contexte du projet, ses enjeux pour l'entreprise Engie et la problématique scientifique qu'il pose. Dans un second temps nous décrivons l'organisation du travail et les moyens mis en oeuvre pour y répondre. La troisième partie propose un état de l'art des méthodes existant dans les publications scientifiques similaire à notre problématique. Par la suite nous décrivons l'approche méthodologique que nous avons choisie de mettre en oeuvre, dont chaque étape sera détaillée dans le chapitre "Réalisation du projet", après une section consacrée à la description des données. Dans la dernière partie de ce rapport, nous proposerons une synthèse des résultats.

\textbf{Note} : compte tenu de l'état d'avancement du projet, la majeure partie du chapitre concernant la réalisation et les résultats obtenus comporte des blancs qui seront complétés dans les mois à venir.