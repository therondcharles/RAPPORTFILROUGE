
\chapter{Decomposition en ondelettes}



\section{Principe}
\paragraph{}
Pour le traitement du signal, différentes de méthodes de décomposition existent.
La plus connue et utilisée, est la transformée de Fourier. Cette dernière permet de filtrer le signal à différentes fréquences temporelles. Elle informe donc sur le contenu fréquentiel du signal.
\paragraph{}
Elle possède une limitation majeure dans la connaissance du comportement local d'une fonction.
Ce défaut a été observé par D.Gabor  ~\cite{}
Elle n'est donc pas adéquate pour l'étude de signaux dans un régime transitoire.
\paragraph{}
La décomposition en ondelettes permet de pallier à ce défaut, elle permet en effet de prendre en compte la variabilité fréquentielle de la fonction. 
Elle donne une \emph{représentation temps-fréquence d'un signal} ~\cite{JBigot}.
et permet donc de fournir une localisation temporelle.

Fourier  wavenumber 
Ondelette  temps et wab
\subsection{Definition}
La transformée en ondelettes ~\cite{JBigot} continue d'une fonction $f  \in L^{2}(\R)$ au point $x \in \R $
et à l'échelle $ s>0 $ est définie par  :
\begin{equation}
W_{f}(x,s) = <f,\Psi_{x,s}> = \int^{+\infty}_{- \infty} \frac{1}{\sqrt{s}}f(t)\Psi(\frac{t-x}{s})dt
\end{equation} 
Une transformée en ondelettes peut s'écrire sous la forme d'un filtrage par convolution :
\begin{equation}
W_{f}(x,s) = f * \psi^{*}_{s}(x)
\end{equation}
avec  $\psi^{*}_{s}(x) = \frac{1}{\sqrt{s}} \psi(\frac{-t}{s})$

Inverse transformation

La transformée en ondelette satisfait les propriétés de conservation de l'énergie du signal.



\paragraph{}
ondelettes

propriété:
support compact, orthogonalité, lissante
\paragraph{}
Type d'ondelettes

Daubechie

\paragraph{}
La transformée en ondelette permet de représenter une fonction comme une combinaison linéaire de
fonctions de bases d'ondelettes.

\paragraph{}
Les coefficients de la transformée en ondelettes sont calculées par

\section{Méthodologie}
\subsection{Decomposition}

\subsection{Détection d'anomalie à l'aide des coefficients dans la base en ondelettes}
Pour détecter les anomalies, on se concentrera sur la projection des données sur la base formée des coefficients d'ondelettes.
\section{Résultat}


\begin{thebibliography}{1}
\bibitem{StMallat}
	  Stéphane Mallat,
	  \emph{A Wavelet Tour of Signal Processing}.
	  Elsevier, 
	  1999.
	  
\bibitem{JBigot}
	Jérémie Bigot,
	\emph{Analyse par ondelettes}
	Université Paul Sabatier,
	2009		  
\bibitem{AGraps}
	Amara Graps
	\emph{An Introduction to Wavelets}
	 IEEE computational science and engineering, 1995, vol. 2, no 2, p. 50-61,
	 1995.
\bibitem{Cchui}
		Charles K Chui
      	\emph{An Introduction to Wavelets}
      	 Elsevier, 2016.
\bibitem{GRama}
	GENÇAY, Ramazan, SELÇUK, Faruk, et WHITCHER, Brandon J
	 \emph{An introduction to wavelets and other filtering methods in finance and economics} 			      Elsevier, 2001.
\bibitem{WojPrz} 
  WOJTASZCZYK, Przemyslaw. 
  \emph{A mathematical introduction to wavelets.}
   Cambridge University Press, 1997.
\bibitem{GStrang}  
	STRANG, Gilbert et NGUYEN, Truong.
	\emph{Wavelets and filter banks.}
	 SIAM, 1996.
\bibitem{SMalla92} 	 
MALLAT, Stephane et HWANG, Wen Liang.
\emph{Singularity detection and processing with wavelets. IEEE transactions on information theory} X vol. 38, no 2, p. 617-643, 1992.   
   
\end{thebibliography}