
\chapter{Détection d'anomalie}

\section{Présentation générale des différentes méthodes}
\subsection{Types d'anomalies}
La détection d'anomalie dans les données est une méthode consistant à trouver des motifs non 
usuels dans la donnée. Ces anomalies sont différentes du bruit existant dans les données.
Il existe différent types d'anomalies:
\begin{enumerate}
\item Les anomalies ponctuelles (un point dont la valeur excède la gamme de valeur de la distribution des données)
\item Les anomalies contextuelles (des valeurs non usuels pour le contexte donné)
\item Les anomalies collectives (une gamme de valeur non usuels)
\end{enumerate}

\paragraph{}
Dans le cas où on suppose que les données sont paramétriques, on suppose
que les données normales suivent une loi gaussienne.
Les anomalies sont les données qui ne suivent pas cette distribution.

L'anormalité est caractérisée par la distance de Mahalanobis
\begin{equation}
d(x,y) = \sqrt{(x-y)^\top \Sigma (x-y)} 
\end{equation}


\paragraph{}
Si on ne suppose rien sur la loi que suit les données normales, d'autres méthodes sont utilisées comme 

\begin{enumerate}
\item Density based
\item Distance based
\end{enumerate}

\subsubsection{Random forest}
On définira une notion de similarité et de distance entre les données .

La similarité entre les 2 observation est basé sur le nombre de fois les 2 observations
correspondent à la même feuille. 
\begin{equation}
prox(n,k)= \frac{similarity}{number tree}
\end{equation}
Les anomalies correspondent aux observations qui ont une petite pro par rapport aux autres observations 

%\subsubsection{Contextual anomalies}
\subsection{Isolation Forest}
Un arbre est construit pour détecter l'anomalie




\subsection{LOF (Local Outlier Factor)}
Identifier la densité local de chaque point et la comparer avec les voisins.
The local reachibility factor density is computed as

\begin{equation}
density  = ..
\end{equation}
\paragraph{}


\subsection{1 class SVM}


\subsection{L'algorithme du LSTM}

\subsection{Extraction des coefficients des ondelettes à utiliser comme feature}
Les données peuvent être projetés dans une autre base pour pouvoir détecter le comportement normal et anormal.
Dans le cas des séries temporelles, on utilise souvent une base composée des coefficients d'ondelettes.

%Il existe une littérature assez conséquente sur la détection d'anomalie pour les maladies cardiaques. Dans ce domaine, on utili


\begin{thebibliography}{1}
\bibitem{Breiman}
BREIMAN, Leo. 
\emph{Random forests.}
Machine learning, 2001, vol. 45, no 1, p. 5-32.

\bibitem{BreunigM}
	 BREUNIG, Markus M., KRIEGEL, Hans-Peter, NG, Raymond T., et al. 
	 \emph{LOF: identifying density-based local outliers.} 
	 In : ACM sigmod record. ACM, 2000. p. 93-104.
	  
\bibitem{ChandolaV}	  
	  CHANDOLA, Varun, BANERJEE, Arindam, et KUMAR, Vipin. 
	  \emph{Anomaly detection: A survey.}
	   ACM computing surveys (CSUR), 2009, vol. 41, no 3, p. 15.
	   
\bibitem{MalhotraP}	  
	MALHOTRA, Pankaj, VIG, Lovekesh, SHROFF, Gautam, et al.
	\emph{Long short term memory networks for anomaly detection in time series.} 
	In : Proceedings. Presses universitaires de Louvain, 2015. p. 89.
	   
\bibitem{KeohgE}    
   KEOGH, Eamonn, LONARDI, Stefano, et CHIU, Bill'Yuan-chi'. 
   \emph{Finding surprising patterns in a time series database in linear time and space.} In : Proceedings of the eighth ACM SIGKDD international conference on Knowledge discovery and data mining. ACM, 2002. p. 550-556.
	   
\bibitem{LiZ} 	
Li, Z., Li, Z., Yu, N., Wen, S. 	   
	   \emph{Locality-Based Visual Outlier Detection Algorithm for Time Series}
	   Security and Communication Networks, 2017.  
\bibitem{ChenC} 	   
	Chen, C., Liu, L. M. (1993).
\emph{Joint estimation of model parameters and outlier effects in time series.}
Journal of the American Statistical Association, 88(421), 284-297.	   

\bibitem{DingH} 	
DING, Hui, TRAJCEVSKI, Goce, SCHEUERMANN, Peter, et al. 
\emph{Querying and mining of time series data: experimental comparison of representations and distance measures.}
 Proceedings of the VLDB Endowment, 2008, vol. 1, no 2, p. 1542-1552.	
	

	
	   
\end{thebibliography}

