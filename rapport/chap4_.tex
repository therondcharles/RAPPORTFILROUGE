
\chapter{Decomposition en ondelettes}



\section{Principe}
\paragraph{}
Pour le traitement du signal, différentes méthodes de décomposition existent.
La plus connue et utilisée, est la transformée de Fourier. Cette dernière permet de filtrer le signal à différentes fréquences temporelles. Elle informe donc sur le contenu fréquentiel du signal.
\paragraph{}
Elle possède une limitation majeure dans la connaissance du comportement local d'une fonction.
Ce défaut a été observé par D.Gabor qui a réussi à pallier à cette inconvénient en utilisant la STFT (short-time Fourier transform). Seulement la résolution temporelle restait insuffisante.
Elle n'est donc pas adéquate pour l'étude de signaux dans un régime transitoire. ~\cite{GaoR}
\paragraph{}
La décomposition en ondelettes permet de pallier à ce défaut.
Elle prend en compte la variabilité fréquentielle de la fonction. 
Elle donne donc une \emph{représentation temps-fréquence d'un signal} ~\cite{JBigot}.
et permet donc de fournir une localisation temporelle.

\subsection{Définition}
La transformée en ondelettes ~\cite{JBigot} continue d'une fonction $f  \in L^{2}(\R)$ au point $x \in \R $
et à l'échelle $ s>0 $ est définie par  :
\begin{equation}
W_{f}(x,s) = <f,\Psi_{x,s}> = \int^{+\infty}_{- \infty} \frac{1}{\sqrt{s}}f(t)\Psi(\frac{t-x}{s})dt
\end{equation} 
Une transformée en ondelettes peut s'écrire sous la forme d'un filtrage par convolution :
\begin{equation}
W_{f}(x,s) = f * \psi^{*}_{s}(x)
\end{equation}
avec  $\psi^{*}_{s}(x) = \frac{1}{\sqrt{s}} \psi(\frac{-t}{s})$

\paragraph{}
La transformée en ondelette satisfait les propriétés de conservation de l'énergie du signal.



\paragraph{}
L'ondelette choisie doit avoir :
\begin{enumerate}
\item un support compact pour obtenir une localisation dans l'espace
\item une moyenne égale à 0


\end{enumerate}
propriété:
 orthogonalité, lissante
\paragraph{}
Il existe plusieurs types d'ondelettes :
\begin{enumerate}
\item Haar
\item Mexican hat
\item Morlet
\item Daubechie

\end{enumerate}

On utilisera l'ondelette Daubechie qui est la plus souvent utilisée d'après la littérature.

\paragraph{}
La transformée en ondelette permet de représenter une fonction comme une combinaison linéaire de fonctions de bases d'ondelettes.

Elle permet de réprésenter une fonction dans une \emph{représentation multirésolution d'ondelettes}. Celà consiste à une séquences d'espaces fermées $V_{m}$ de $ L^{2}(\R)$ où
$L^{2}(\R)$ est un espace d'Hilbert de outes les fonctions au carré intégrables.
Ces espaces caractérisent le comportement d'une fonction à l'échelle $2^{m}$ échantillons par unité de longueurs.


\paragraph{}
Les coefficients de la transformée en ondelettes sont calculées par
\begin{equation}
W_{f}(x,s) = f * \psi^{*}_{s}(x)
\end{equation}

\section{Méthodologie}

\subsection{Decomposition}
La fonction que l'on souhaite décomposer est réalisée de la façon suivante :
\begin{equation}
f(x)= \sum^{2j0-1}_{k=0} \alpha_{0,k}\phi_{j0,k} +\sum \sum^{2j0-1}_{k=0}\beta_{j,k}\psi_{j,k}
\end{equation}
%\alpha_{0,k}\phi_{j0,k} +\sum \sum^{2j0-1}_{k=0}\beta_{j,k}\psi_{j,k}
\begin{equation}
f(x) =f_{J0}(x) +\sum_{j>j0} D_{j}(x)
\end{equation}

Les coefficients d'ondelettes sont calculées de la manière suivante :
avec le coefficient d'échelle :
\begin{equation}
\alpha = \int^{1}_{0}f(x)\phi(x)dx
\end{equation}
et le coefficient de détail :
\begin{equation}
\beta_{j,k} = \int^{1}_{0}f(x)\psi(x)dx
\end{equation}

$phi(x)$ est appelé l'ondelette mère
et $\psi(x) $ est l'ondelette père .

Pour une ondelette de Haar (ou Daubechie) , elles sont de la forme 



\begin{equation}
\psi(x) = \left\{
    \begin{array}{ll}
        -1 & \mbox{si } \{x\} \in [0,1/2] \\
        1 & \mbox{si } \{x\} \in ]1/2,1]
    \end{array}
\right.
\end{equation}


\begin{equation}
\phi(x) = \left\{
    \begin{array}{ll}
        1 & \mbox{si } \{x\} \in [0,1[ \\
        0 & \mbox{sinon.} 
     \end{array}
\right.
\end{equation}



\paragraph{}
Débruitage par seuillage
\subsection{Détection d'anomalie à l'aide des coefficients dans la base en ondelettes}
Pour détecter les anomalies, on se concentrera sur la projection des données sur la base formée des coefficients d'ondelettes.
\section{Résultat}


\begin{thebibliography}{1}
\bibitem{StMallat}
	  Stéphane Mallat,
	  \emph{A Wavelet Tour of Signal Processing}.
	  Elsevier, 
	  1999.
	  
\bibitem{JBigot}
	Jérémie Bigot,
	\emph{Analyse par ondelettes}
	Université Paul Sabatier,
	2009		  
\bibitem{AGraps}
	Amara Graps
	\emph{An Introduction to Wavelets}
	 IEEE computational science and engineering, 1995, vol. 2, no 2, p. 50-61,
	 1995.
\bibitem{Cchui}
		Charles K Chui
      	\emph{An Introduction to Wavelets}
      	 Elsevier, 2016.
\bibitem{GRama}
	GENÇAY, Ramazan, SELÇUK, Faruk, et WHITCHER, Brandon J
	 \emph{An introduction to wavelets and other filtering methods in finance and economics} 			      Elsevier, 2001.
\bibitem{WojPrz} 
  WOJTASZCZYK, Przemyslaw. 
  \emph{A mathematical introduction to wavelets.}
   Cambridge University Press, 1997.
\bibitem{GStrang}  
	STRANG, Gilbert et NGUYEN, Truong.
	\emph{Wavelets and filter banks.}
	 SIAM, 1996.
\bibitem{SMalla92} 	 
MALLAT, Stephane et HWANG, Wen Liang.
\emph{Singularity detection and processing with wavelets. IEEE transactions on information theory} X vol. 38, no 2, p. 617-643, 1992.   

\bibitem{GaoR}  
Gao, R. X., Yan, R. (2010)
\emph{Wavelets: Theory and applications for manufacturing.} 
Springer Science \& Business Media.   
%   
\bibitem{Guler}
GULER, Inan et UBEYLI, Elif Derya
\emph{ECG beat classifier designed by combined neural network model.}
 Pattern recognition, 2005, vol. 38, no 2, p. 199-208.
\bibitem{INCE}
INCE, Turker, KIRANYAZ, Serkan, et GABBOUJ, Moncef.
\emph{A generic and robust system for automated patient-specific classification of ECG signals.} IEEE Transactions on Biomedical Engineering, 2009, vol. 56, no 5, p. 1415-1426.

\end{thebibliography}