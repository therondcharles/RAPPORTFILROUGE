


\chapter{Présentation du projet}

\subsection{Presentation de ENGIE}

\subsection{But}
A partir des données de signaux électriques en provenance des panneaux solaires, il s’agit de détecter les sous-performances des équipements et d’analyser les empreintes des différentes anomalies. Ces informations seront transmises à l’exploitant. Dans un second temps, selon l’avancement, il faudrait aussi identifier les causes des perturbations.
Surveiller nos équipements et détecter les anomalies qui illustrent la sous performance des équipements.


\section{Introduction scientifique}

\section{Etude bibliographique sur la détection d'anomalies pour un système transient}
En analysant la littérature concernant 
\subsection{LSTM}

\subsection{Décomposition en Wavelet + réseau de neuronne}

\subsection{Random forest}





\section{Methodologie}

\begin{enumerate}
\item Analyser les données et décomposer les résidus pour retrouver les différents effets
\begin{itemize}
\item  Approche par ondelettes : décomposition du signal sur la base de signatures typiques du signal
\end{itemize}
\item Approche globale pour retrouver les effets de façon simultanée
\end{enumerate}

%\bibliography{}

%xarray (numpy et pandas)
